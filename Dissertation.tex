%>>> Welcome to Vanderbilt's Dissertation Template <<<%
%>>> Prepared by J. Zachary Gaslowitz in 2018 <<<%

%>>> Pick your font size ('10pt', '11pt', or '12pt') <<<%
%>>> When you're ready, change 'signatures' to 'approved' <<<%
\documentclass[final, 12pt, signatures]{vuthesis}
%>>> Some options: <<<%
% draft vs final, for packages that should behave differently in the final version
% 10pt vs 11pt vs 12pt, base font size
% signatures vs approved: how committee is shown on title page
% smallnames vs largenames: size of committee names under signature lines
% centersec vs flushsec: alignment of section headings
% centersubsec vs flushsubsec: alignment of subsection headings

%>>> If you'd like, you can customize the various titles and headings <<<%
\titlesize{\large}
\chaptersize[\large]{\Large}
\sectionsize{\large}


%>>> Definitely delete this; you don't need any filler text in your beautiful dissertation <<<%
\usepackage{blindtext}

%>>> Some useful packages.  Uncomment whichever you'd like, or add your own! <<<%

% For creating a list of abbreviations.
% If you don't want one, delete these lines and `\printnomenclature' below
\usepackage[intoc]{nomencl}
	\renewcommand{\nomname}{LIST OF ABBREVIATIONS}
	\makenomenclature

% Good math packages:
\usepackage{amsmath, amsfonts, amssymb, mathtools} % many nice math commands and environments
%\usepackage{amsbsy, bm} % useful for bold math symbols
\usepackage{amsthm} % Some useful environments for theorems, the likes
	% Uses the same counter for all of these environments, resets numbering for each chapter
	\newtheorem{theorem}{Theorem}[chapter]
	\newtheorem{lemma}[theorem]{Lemma}
	\newtheorem{proposition}[theorem]{Proposition}
	\newtheorem{corollary}[theorem]{Corollary}
	\newtheorem{conjecture}[theorem]{Conjecture}
	\newtheorem{observation}[theorem]{Observation}
	\newtheorem{claim}{Claim}[theorem] % (numbered inside their theorem/lemma)
	\theoremstyle{definition} 
	\newtheorem{definition}[theorem]{Definition}

% Graphics packages:
%\usepackage{tikz} % A great and powerful library for drawing pictures
%\usepackage{graphicx} % For including external images
	%\graphicspath{{images/}} % Look for images in this directory
	%\DeclareGraphicsExtensions{.pdf,.jpeg,.png,.PNG,.eps,.tiff}
%\usepackage{rotating} % Good for creating rotated figures and tables

%\usepackage{algorithm} % Creates a new float for algorithms
%\usepackage{algpseudocode} % Facilitates the typing of pseudocode


\usepackage[notref]{showkeys} % in draft mode, shows labels in document
\usepackage{url} % improved typesetting of urls with \url{}
\usepackage{subcaption} % allows for subfigures and subtables
\usepackage{nowidow} % Discourages pagebreaks at the first or last line of a paragraph

%!! Most packages should be loaded before hyperref !!%
\usepackage{hyperref} % automatically creates links where you'd want them to be

%>>> You can set margins here, among other things <<<%
\usepackage[top=1in, bottom=1in, inner=1.25in, outer=1.25in]{geometry}



%>>> Choose your line spacing. <<<%
%\singlespacing
%\onehalfspacing
\doublespacing



%>>> Tell us a little bit about yourself. <<<%
\title{A Compendium of Super Advanced Mathematics Which Will Undoubtedly Improve Every Aspect of the Field}
\author{J. Zachary Gaslowitz}
\degree{Doctor of Philosophy}
\major{Mathematics}
\graduationdate{May 11, 2018}
\oncommittee{Paul Erd\"os, Ph.D.}
\oncommittee{Lise Meitner, Ph.D.}
\oncommittee{Richard Feynman, Ph.D.}
\oncommittee{Emmy Noether, Ph.D.}
\oncommittee{Marie Curie, Ph.D.}
\oncommittee{Brian May, Ph.D.}



\begin{document}

\maketitle

\preliminaries

%>>> (You can remove this section if you don't want it) <<<%
\copyrightpage{2018}

%>>> (You can remove this section if you don't want it) <<<%
\begin{dedication}
	This template is dedicated to those people who are awesome.
\end{dedication}

%>>> (You can remove this section if you don't want it) <<<%
\acknowledgements
I would like to thank my advisor and the federal government for their intellectual and financial support, respectively.
Without either of these parties, I'm sure that I wouldn't have a dissertation in which to write these acknowledgments.

\blindtext

%>>> (You can remove this section if you don't want it) <<<%
\preface
\blindtext

\tableofcontents
\listoffigures
\listoftables

%>>> (You can remove this section if you don't want it) <<<%
\printnomenclature

\maincontent
%>>> I highly suggest putting each chapter in its own file <<<%

\chapter{A Very Important Thing}

\blindtext

\section{The part where we prove something exciting}

\subsection{Some background to get things started}

\begin{figure}
	\centering
	\emptybox{4in}{2in}
	\caption{A sample figure}
\end{figure}

\blindtext

\blindtext

\subsection{Where things really get interesting}

\blindtext

\begin{figure}
	\centering
	\begin{subfigure}{.5\textwidth}
		\centering
		\emptybox{2in}{1in}
		\caption{The first subfigure}
	\end{subfigure}%
	\begin{subfigure}{.5\textwidth}
		\centering
		\emptybox{2in}{1in}
		\caption{The second subfigure}
	\end{subfigure}
	\caption{Another figure}
\end{figure}

\begin{table}
	\centering
	\begin{tabular}{r|lc}
		Name    & Favorite Color & Age \\ \hline
		Albert  & Yellow         & 23 \\
		Beth    & Violet         & 30 \\
		Charles & Transparent    & 83 \\
		Dorothy & Ruby-Red       & 16
	\end{tabular}
	\caption{A particularly important table} \label{table:PeopleOfInterest}
\end{table}

\begin{theorem}
	For every number $n$, there exists some a number $m$ such that $m > n$.
\end{theorem}

\begin{proof}
	Let $n$ be any number.
	\begin{claim}
		Addition of real numbers is well defined.
	\end{claim}
	\begin{claim}
		The number one exists and is greater than zero.
	\end{claim}
	The theorem follows from the above claims by letting $m$ be $n+1$.
\end{proof}

\blindtext

\section{Discussing the importance of this work}

\blindtext[2]

\chapter{A Very Long Title Which Explains In Too Much Detail The Subject Matter At Hand}

\section{The part where we prove something exciting}

\blindtext

\subsection{Some background to get things started which is also very long, despite not conveying much information}

\blindtext

\begin{figure}
	\centering
	\emptybox{3.5in}{2in}
	\caption{A sample figure}
\end{figure}

\blindtext[2]

\begin{table}
	\centering
	\begin{tabular}{r|lc}
		Name    & Favorite Color & Age \\ \hline
		Albert  & Yellow         & 23 \\
		Beth    & Violet         & 30 \\
		Charles & Transparent    & 83 \\
		Dorothy & Ruby-Red       & 16
	\end{tabular}
	\caption{A particularly important table}
\end{table}

\subsection{Where things really get interesting}

\blindtext

\begin{table}
	\centering
	\begin{tabular}{r|lc}
		Name    & Favorite Color & Age \\ \hline
		Albert  & Yellow         & 23 \\
		Beth    & Violet         & 30 \\
		Charles & Transparent    & 83 \\
		Dorothy & Ruby-Red       & 16
	\end{tabular}
	\caption{A significantly less important table, but one which inexplicably exists none-the-less}
\end{table}

\blindtext

\section{Discussing the importance of this work}

\blindtext[2]


\appendix
%>>> Any chapter after this line will be typeset as an appendix <<<%

\chapter{Previously Excluded Details}
\blindtext[2]

\chapter{Unnecessary Information}
\section{A Discussion of the Nature of Necessity}
\blindtext
\section{The Really Worthless Part}
\blindtext

\nomenclature{$c$}{Speed of light in a vacuum inertial system}
\nomenclature{$h$}{Planck Constant}
\nomenclature{$g$}{Gravitational Constant}
\nomenclature{$\mathbb{R}$}{Real Numbers}
\nomenclature{$\mathbb{C}$}{Complex Numbers}
\nomenclature{$\mathbb{H}$}{Quaternions}
\nomenclature{$\rho$}{Friction Index}



%\ForEach{,}%
%{%
%	\addtocounter{chapter}{1}%
%	\addcontentsline{toc}{chapter}{\protect\numberline{\thechapter}Another chapter}%
%	\ForEachX{;}%
%	{%
%		\ifnum\thislevelcount=1%
%		\else%
%			\addtocounter{section}{1}%
%			\addcontentsline{toc}{section}{\protect\numberline{\thesection}Another section}%
%		\fi
%	}
%	{\thislevelitem}
%}%
%{;;;,;,,}

\nocite{*}

%>>> You may need to change the bibliography style, depending on your field. <<<%
\bibliographystyle{amsplain} 
\bibliography{Dissertation}


\end{document}
