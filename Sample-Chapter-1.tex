\chapter{A Very Important Thing}

\blindtext

\section{The part where we prove something exciting}

\subsection{Some background to get things started}

\begin{figure}
	\centering
	\emptybox{4in}{2in}
	\caption{A sample figure}
\end{figure}

\blindtext

\blindtext

\subsection{Where things really get interesting}

\blindtext

\begin{figure}
	\centering
	\begin{subfigure}{.5\textwidth}
		\centering
		\emptybox{2in}{1in}
		\caption{The first subfigure}
	\end{subfigure}%
	\begin{subfigure}{.5\textwidth}
		\centering
		\emptybox{2in}{1in}
		\caption{The second subfigure}
	\end{subfigure}
	\caption{Another figure}
\end{figure}

\begin{table}
	\centering
	\begin{tabular}{r|lc}
		Name    & Favorite Color & Age \\ \hline
		Albert  & Yellow         & 23 \\
		Beth    & Violet         & 30 \\
		Charles & Transparent    & 83 \\
		Dorothy & Ruby-Red       & 16
	\end{tabular}
	\caption{A particularly important table} \label{table:PeopleOfInterest}
\end{table}

\begin{theorem}
	For every number $n$, there exists some a number $m$ such that $m > n$.
\end{theorem}

\begin{proof}
	Let $n$ be any number.
	\begin{claim}
		Addition of real numbers is well defined.
	\end{claim}
	\begin{claim}
		The number one exists and is greater than zero.
	\end{claim}
	The theorem follows from the above claims by letting $m$ be $n+1$.
\end{proof}

\blindtext

\section{Discussing the importance of this work}

\blindtext[2]
